% ********** Rozdział 4 **********
\chapter{Podsumowanie}
\section{Podsumowanie projektu}

Podsumowując projekt "System Zarządzania Symulacją Samochodu", można stwierdzić, że zostały zrealizowane główne założenia projektu, które obejmowały stworzenie realistycznego doświadczenia jazdy samochodem wraz z możliwością interaktywnego sterowania pojazdem oraz monitorowania kluczowych parametrów. System został zaprojektowany w sposób prosty, czytelny i łatwy dla użytkownika, co pozwoliło na płynną interakcję i maksymalne zrozumienie procesu symulacji. W ramach projektu opracowano również diagram klas, który przedstawił strukturę klas i ich relacje w systemie. Wykorzystano technologie .NET Framework oraz narzędzia takie jak Visual Studio do implementacji projektu. Repozytorium znajduje się na platformie GitHub, co umożliwia monitorowanie kodu. Warstwa użytkowa projektu została zaprezentowana poprzez interaktywną konsolę, która umożliwiała wykonywanie różnych akcji związanych z symulacją samochodu, takich jak włączenie silnika, przyśpieszanie, zmiana biegów, hamowanie, a także monitorowanie parametrów pojazdu, takich jak prędkość, obroty silnika czy stan paliwa. W ramach podsumowania projektu można stwierdzić, że udało się zrealizować postawione cele i założenia, tworząc kompleksowy system do symulacji i analizy zachowania samochodu. 


% ********** Koniec rozdziału **********
