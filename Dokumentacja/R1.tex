% ********** Rozdział 1 **********
\chapter{Opis założeń projektu}
\section{Założenia projektu}
\begin{itemize}
\item 
Projekt zakłada stworzenie realistycznego doświadczenia jazdy samochodem, umożliwiając jednocześnie symulację podstawowych funkcji pojazdu oraz monitorowanie kluczowych parametrów. Użytkownik będzie mógł interaktywnie kontrolować samochód, włączając i wyłączając silnik, zmieniając biegi, przyspieszając i hamując. Podczas symulacji będzie mógł również obserwować różne parametry pojazdu, takie jak aktualna prędkość, aktualny bieg i obroty silnika. Ponadto, program będzie obsługiwał sytuacje, w których jazda nie będzie możliwa ze względu na brak paliwa lub zatarty silnik. 
\end{itemize}

\section{Wymagania funkcjonalne i niefunkcjonalne}
\noindent{Wymagania funkcjonalne:}
\begin{itemize}
\item Symulacja działania samochodu: System będzie umożliwiał symulację różnych aspektów działania samochodu.
\item Monitorowanie zużycia paliwa: System umożliwia monitorowanie ilości zużytego paliwa w czasie rzeczywistym podczas symulacji.
\item Analiza efektywności spalania: System dostarcza analize efektywności spalania w różnych warunkach jazdy na podstawie zebranych danych.
\item Analiza kosztów: System przeprowadza analizę kosztów paliwa na podstawie zebranych danych symulacyjnych.
\item Obserwacja średniej prędkości podróży: System umożliwia obserwację średniej prędkości podróży i dostarczanie statystyk na temat czasu przebytej trasy.
\end{itemize}
\noindent{Wymagania niefunkcjonalne:}
\begin{itemize}
\item Wydajność: Aplikacja działa płynnie i dostarcza informacje w czasie rzeczywistym, zapewniając szybkie i efektywne działanie symulacji samochodu.
\item Intuicyjny interfejs: Interfejs aplikacji jest intuicyjny i łatwy w obsłudze, umożliwiając użytkownikowi wygodne korzystanie z funkcji symulacji samochodu.
\item Dostępność systemu: System zapewnia wysoką dostępność, umożliwiając korzystanie z funkcji symulacji samochodu przez większość czasu. 
\end{itemize}


% ********** Koniec rozdziału **********
